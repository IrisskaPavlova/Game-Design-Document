\documentclass{article}
\usepackage{graphicx} 
\usepackage[T2A]{fontenc}
\usepackage{tabularx}


\usepackage{amsmath}

\title{Game Design Document игры "Магия леса"}
\author{
Гарин Андрей\footnote{Research University Higher School of Economics Nizhny Novgorod} \\
Павлова Ирина \\
Дрямин Даниил \\
Калетурина Полина \\
Курятников Дмитрий \\
Мамонтов Игорь
}
\date{Ноябрь 2024}

\begin{document}

\maketitle

\tableofcontents
\newpage

\section{Введение}

Game Design Document (GDD) — это важный инструмент для разработчиков
игр, который помогает структурировать и организовать все аспекты игры.
GDD служит руководством для всей команды, обеспечивая единое видение
проекта. В этом документе описываются основные механики, сюжет, персонажи,
уровни и другие элементы игры. Создание GDD позволяет избежать множества
проблем на этапе разработки, таких как недопонимание между членами
команды или несоответствие конечного продукта первоначальной идее.

\section{Концепция}
\subsection{Введение}
В мире игры "Магия леса" магия пронизывает каждую частицу природы.
Игрок — могучий маг, исследующий разнообразные
локации. Каждый враг обладает своими магическими умениями и стратегиями,
что требует от игрока применения хитроумных тактик.
Сражаясь с помощью заклинаний, магу предстоит
сразиться с опасными противниками и преодолеть множество трудностей,где
главным испытанием станет бой с Драконом. Это величественное и устрашающее
существо будет не только испытанием физической силы и мастерства, но
и проверкой стратегического мышления Игрока. Чтобы одержать победу,
ему придется изучить повадки дракона, найти слабые места и использовать
все доступные ресурсы. Этот эпический
поединок станет кульминацией его приключений и определит, сможет ли он
стать настоящим героем, или будет навсегда забыт в легендах.
В этом волшебном мире, где все пронизано магией, его судьба кроется в его собственных руках.


\subsection{Жанр и аудитория}
\begin{itemize}
    \item Жанр: 
    Платформер
    \item Возрастное ограничение:
    12+ (в игре могут присутствовать элементы фэнтезийного насилия, но без излишней жестокости и с акцентом на приключения)
    \item Аудитория:
    Игра привлечет любителей приключений и фэнтези, ценящих богатый и детализированный игровой мир, интересный сюжет и сложные игровые механики. Возрастной диапазон игроков варьируется от 12 до 35 лет, так как игра сочетает в себе доступность и глубину геймплея. Элементы исследования обеспечивают высокий уровень вовлеченности.
\end{itemize}

\subsection{Основные особенности игры}
В игре реализованы механики атаки и уворота. Чтобы зарядить атаку, нужно находиться на земле, во время зарядки атаки магическим шаром за игроком остаётся полный контроль движения персонажа. Также на последнем уровне игрока ожидает босс (дракон, который стреляет огненными шарами).
Примерное время прохождения - 10 минут.

\subsection{Описание игры}
Игра-платформер, созданная на Unity, с пятью уровнями. Игроки проходят уровни, побеждают врагов и избегают препятствий. На каждом уровне, кроме последнего, есть телепорт, который перемещает игрока на следующий уровень. Главное меню предлагает начать новую игру, загрузить последнее сохранение или выйти. Герой обладает магическими способностями и навыком телепорта, что помогает в сражениях с монстрами и прохождении локаций. В игре присутствуют три вида локаций: летняя, осенняя и зимняя. 

\subsection{Предпосылки создания}
 Разработчики вдохновлялись тематикой "фентези" и такими произведениями как "Властелин колец", "Хоббит", а также играми схожего жанра.
\subsection{Платформа}
%\newpage
%\begin{table}
\begin{tabularx}{\textwidth}{|X|X|X|} \hline  
    Требования&  Минимальные& Рекомедуемые\\ \hline  
    ОС&  Windows 8& Windows 10\\ \hline  
    Прцессор&  Intel Core 2 Duo E4400& Intel Core i3-1000NG4\\ \hline  
    ОЗУ&  2GB& 4GB\\ \hline  
    Сводобное место на HDD&  3GB& 4GB\\ \hline  
    Видеокарта&   GeForce 210; RADEON X600  HyperMemory&  GeForce GT 230; Radeon HD 6550D\\ \hline  
    VRAM&  256MB& 512MB\\ \hline  
    DirectX&  9.0& 9.0\\ \hline  
    Управление&  Клавиатура, мышь& Клавиатура, мышь\\ \hline 
\end{tabularx}    
%\end{table}


\newpage
\section{Функциональная спецификация}
\subsection{Принципы игры}
\subsubsection{Суть игрового процесса}
Игрок будет сражаться с противниками посредством магии, избегать ловушки с помощью соспобности телепортации и перепрыгивать через пропасти. В конце игры находится могущественный босс - дракон, с большим количеством жизней.

\subsubsection{Ход игры и сюжет}
% Ход игры и сюжет здесь...

\subsection{Физическая модель}
% Описание физической модели здесь...

\subsection{Персонаж игрока}
Персонаж игрока - могущественный маг Гэндальф. 
Маг облачен в светло-фиолетовый камзол, его голову прикрывает фиолетовый капюшон, а на ногах красуются сапоги, сделанные из кожи злой живой ели, побежденной в неравном бою. 
В правой руке он держит посох с магическим синим камнем, являющимся источником магии и указывающим на его принадлежность к магам-защитникам леса. 
Над левой рукой Гэндальфа парит книга с заклинаниями, полученная от Создателей. Гэндальф - бесстрашный, отважный  и могущественный маг, стремящийся навести порядок в Магических Лесах, тем самым выполнив свое предназначение. 
Он знает сильнейшие из всех, какими могут обладать лесные маги, заклинания: Шаровой удар и Кротовая телепортация.  
В процессе игры Гэндальф не раз проявит свое мастерство и отвагу, чтобы избавить Магические Леса от тирании древнего дракона и вернуть мир в эти прекрасные земли.  

\subsection{Элементы игры}
Гэндальф - главный герой, обладает способностями телепортации и стрельбы из посоха.
Пешие враги - враги, бегающие по определенной зоне, столкновение с ними снимает одну жизнь, также они могут столкнуть игрока с платформы. Они имеют 3 жизни.
Летающие враги - враги, патрулирующие территорию по воздуху, бессмертны, снимают игроку одну жизнь при столкновении с ним.
Шипы - ловушки, снимают игроку одну жизнь при столкновении с ним.
Дракон - главный босс, имеет 20 жизней, стреляет огнем в направление игрока и перемещается как пеший враг.

\subsection{Искусственный интеллект}
Рядовые враги имеют простой интеллект, единственное его использование - патрулирование территорий. Босс дракон имеет более сложный искусственный интелект - он стреляет снарядами в направление персонажа.

\subsection{Многопользовательский режим}
% Многопользовательский режим здесь...

\subsection{Интерфейс пользователя}
\subsubsection{Блок-схема}
        \centering
        \includegraphics[width=0.5\linewidth]{diagram (2).png}
        \label{fig:enter-label}
  

\subsubsection{Функциональное описание и управление}
% Описание управления здесь...

\subsubsection{Объекты интерфейса пользователя}
Счетчик количетсва жизней персонажа.

\subsection{Графика и видео}
\subsubsection{Общее описание}
2D графика. Игра сделана в стилистике фентези, доминируют яркие цвета, такие как зеленый и фиолетовый.
\subsubsection{Двумерная графика и анимация}
Интерфейс состоит из счетчика жизней персонажа. Вся графика игры выполненна в формате пиксель арт.
\subsection{Звуки и музыка}
\subsubsection{Общее описание}
Музыкальное оформление игры построено на контрасте между спокойными, мелодичными мотивами и напряжёнными, динамичными композициями. В начале игры, когда игрок проникает в мир волшебного леса, звучит таинственная и сказочная музыка, подчеркивающая красоту и загадочность окружающего мира. Эти музыкальные темы способствуют созданию ощущения исследования и приключения, погружая игрока в атмосферу волшебства и открытий.

Однако по мере продвижения к последнему уровню, где волшебнику предстоит сразиться с драконом, музыкальная палитра переходит к напряжённым и тревожным мелодиям. Они создают эффект надвигающейся угрозы и непредсказуемости, подготавливая игрока к кульминационному моменту сражения. Эти музыкальные темы акцентируют важность и страх перед битвой, подчеркивая опасность, которую представляет дракон.

Мир игры будет динамично реагировать на действия игрока. Каждое заклинание, удар или взаимодействие с окружающей средой будет сопровождаться соответствующими звуковыми эффектами, что придаст весомость каждому действию. Дракон, готовясь к нападению, будет издавать угрожающее рычание, погружая игрока в атмосферу страха и волнения. Анимации и действия персонажа будут синхронизированы с музыкой и звуковыми эффектами, что создаст гармоничное и захватывающее впечатление от игрового процесса.
\subsubsection{Звуки и звуковые эффекты}
В Интерфейсе есть навигационные звуки: 
сигналы при наведении курсора на кнопки, переключении вкладок и выполнении действий интерфейса.
\begin{itemize}
  \item Звук нажатия кнопки (легкое «щелчок»).
  \item Звук наведения мыши (мягкий «плёск»).
\end{itemize}
Для спецэффектов использованы эффекты окружения: 
звуки, создающие атмосферу игрового мира.
\begin{itemize}
    \item Звук ветра, проносящегося через лес.
    \item Далёкие звуки боёв и шторма на фоне.
\end{itemize}
А также эффекты магии и умений: 
каждый тип магического умения имеет свой уникальный звук.
\begin{itemize}
    \item Для огненного заклинания — мощный «всплеск огня».
    \item Для ледяного заклинания — тихий «шёпот льда».
\end{itemize}
Для персонажей также добавлены звуковые реакции на действия игрока или окружающие события.
   - Звук шагов (различается по поверхности: камень, трава, грязь).
   - Звуки атаки и получения урона (крики, взрывы).
   


\subsubsection{Музыка}
\begin{itemize}
    \item Главный интерфейс новой игры.
    
При открытии меню главного интерфейса новой игры музыка отсутствует. Это решение позволяет сосредоточиться игрокам на навигации по меню и настройках, не отвлекаясь на фоновые мелодии.
    \item Темы по уровням игры.
    
Музыкальное сопровождение различных уровней игры помогает создать уникальную атмосферу в зависимости от их тематики. Ниже перечислены музыкальные треки, использующиеся в игре:
\begin{itemize}
    \item **Summer in-game.ogg**: Играет в начале игры, где царит спокойная и расслабляющая приключенческая мелодия. Она создает ощущение дружелюбной и красивой летней обстановки, подчеркивая атмосферу исследования.
    \item **Fall in-game.ogg**: Этот трек звучит на уровне с осенней тематикой. Музыка отражает меланхоличную, но в то же время завораживающую атмосферу осеннего леса. Фоновый дизайн уровня также выполнен в осенних тонах, что усиливает эффект от звучащей музыки.
    \item **Bossfight in-game.ogg**: Напряженная мелодия, звучащая во время финальной битвы с драконом. Музыка создает ощущение напряженности и экшена, подчеркивая важность момента и привнося в игру элемент драмы и борьбы.
\end{itemize}
\end{itemize}
\subsection{Описание уровней}
\subsubsection{Общее описание дизайна уровней}
\begin{itemize}
    \item Гланый персонаж - волшебник в фиолетовом плаще, переливающемся на свету, олицетворяет загадочную силу и древнюю мудрость. Его длинный плащ, как будто сотканный из звездной пыли, оставляет за собой легкий шлейф.В руках он держит деревянный посох, усыпанный узорами, с волшебным камнем на вершине, меняющим цвет.Этот образ внушает восхищение и уверенность в присутствии магии.
    \item На певых уровнях царит летняя атмосфера, в которой волшебник начинает свое путешествие из своего маленького домика. Первый уровень насыщен красочными предметами природы, такими как зеленые деревья, яркие яблоки и цветущая трава. Это создаёт ощущение безмятежности и открытости. В этом уровне игрок столкнется с врагами-грибами (мухоморы с ручками и ножками), что добавит элемент забавного противостояния. 
    \item В последующих уровнях царит осенний пейзаж. Здесь игрок увидит, как природа меняется, с теплой цветовой гаммой и опадающими листьями, что придаст уровню меланхоличный, но все еще волшебный вид. 
    \item В финальном уровене появляется огромный красный дракон с рогами и крыльями, олицетворяющий силу и угрозу. Дракон обладает огненными атаками, добавляющими уровень сложности. Этот босс станет кульминацией путешествия волшебника, и его появление подчеркивает важность первой части сюжета, где герой должен преодолеть свои страхи и подготовиться к финальной битве.
\end{itemize}

\subsubsection{Диаграмма взаимного расположения уровней}
\subsubsection{ График введения новых объектов}
\section{Контакты}
\begin{itemize}
    \item Павлова Ирина
    \begin{itemize}
        \item e-mail: iralpavlova@edu.hse.ru 
        \item Телефон: +79200658167 
        \item Ссылка на GitHub: https://github.com/IrisskaPavlova
    \end{itemize}
    \item Андрей Гарин
    \begin{itemize}
        \item e-mail:
        \item Телефон:
        \item Ссылка на GitHub:
    \end{itemize}
    \item Дрямин Даниил
    \begin{itemize}
        \item e-mail:
        \item Телефон:
        \item Ссылка на GitHub:
    \end{itemize}
    \item Калетурина Полина
    \begin{itemize}
        \item e-mail:
        \item Телефон:
        \item Ссылка на GitHub:
    \end{itemize}
    \item Курятников Дмитрий
    \begin{itemize}
        \item e-mail:
        \item Телефон:
        \item Ссылка на GitHub:
    \end{itemize}
    \item Мамонтов Игорь
    \begin{itemize}
        \item e-mail:
        \item Телефон:
        \item Ссылка на GitHub:
    \end{itemize}
  
    
\end{itemize}


\end{document}