\documentclass{article}
\usepackage{graphicx} % Required for inserting images
\usepackage[T2A]{fontenc}
\usepackage{tabularx}


\usepackage{amsmath}

\title{Game Design Document игры "Магия леса"}
\author{
Гарин Андрей\footnote{Research University Higher School of Economics Nizhny Novgorod} \\
Павлова Ирина \\
Дрямин Даниил \\
Калетурина Полина \\
Курятников Дмитрий \\
Мамонтов Игорь
}
\date{Ноябрь 2024}

\begin{document}

\maketitle

\tableofcontents
\newpage

\section{Введение}

Game Design Document (GDD) — это важный инструмент для разработчиков
игр, который помогает структурировать и организовать все аспекты игры.
GDD служит руководством для всей команды, обеспечивая единое видение
проекта. В этом документе описываются основные механики, сюжет, персонажи,
уровни и другие элементы игры. Создание GDD позволяет избежать множества
проблем на этапе разработки, таких как недопонимание между членами
команды или несоответствие конечного продукта первоначальной идее.

\section{Концепция}
\subsection{Введение}
В мире игры "Магия леса" магия пронизывает каждую частицу природы.
Игрок — могучий маг по имени Гендальф, исследующий разнообразные
локации: от густых лесов с таинственными существами до знойных пустынь,
полных древних руин. В каждом шаге его ждет новое открытие — уникальные
растения, способные усиливать его способности, или магические существа,
которые могут стать как опасными противниками, так и ценными союзниками.
В пещерах и подземельях замков его ждут опасные враги и загадочные
артефакты. Каждый враг обладает своими магическими умениями и стратегиями,
что требует от игрока применения хитроумных тактик и комбинаций заклинаний.
Сражаясь с помощью заклинаний и магического оружия, Гендальфу предстоит
сразиться с опасными противниками и преодолеть множество трудностей,где
главным испытанием станет бой с Драконом. Это величественное и устрашающее
существо будет не только испытанием физической силы и мастерства, но
и проверкой стратегического мышления Игрока. Чтобы одержать победу,
ему придется изучить повадки дракона, найти слабые места и использовать
все доступные ресурсы — от магии до мощных артефактов. Этот эпический
поединок станет кульминацией его приключений и определит, сможет ли он
стать настоящим героем, или будет навсегда забыт в легендах.
Но магия — это не только разрушение: игрок также может использовать
её для создания и исцеления. На его пути встречаются пострадавшие деревни,
жители которых нуждаются в помощи, и благодарные селяне могут наградить
его особыми артефактами за его доброту. Разнообразие квестов разнообразит
игровой процесс: поиски затерянных рун, загадочные испытания пользующихся
древней магией, проверяющие его умение ориентироваться в мире, или
даже сложные головоломки, требующие логики и смекалки.
Путешествуя по этому миру, игрок открывает не только секреты магии,
но и свою собственную силу. Он может расширять свои знания, изучая
древние свитки и книги, встречая мудрых наставников и обменяя полученные
знания на новые заклинания. С каждым новым уровнем он погружается
всё глубже в неизведанные глубины магии, и его имя становится легендой,
страхом для врагов и надеждой для друзей.
3
К тому же, в мире существуют знойные пустыни, полные древних руин.
Эти опасные зоны полны ловушек и потенциальных сокровищ, и только
самые смелые решатся исследовать их. Каждый выбор имеет значение:
союзники, способности и магия, которыми он овладел, могут либо привести
к триумфу, либо стать причиной падения в бездну темноты. В этом волшебном
мире, где все пронизано магией, его судьба кроется в его собственных руках.


\subsection{Жанр и аудитория}
\begin{itemize}
    \item Жанр: 
    Платформер
    \item Возрастное ограничение:
    12+ (в игре могут присутствовать элементы фэнтезийного насилия, но без излишней жестокости и с акцентом на приключения)
    \item Аудитория:
    Игра привлечет любителей приключений и фэнтези, ценящих богатый и детализированный игровой мир, интересный сюжет и сложные игровые механики. Возрастной диапазон игроков варьируется от 12 до 35 лет, так как игра сочетает в себе доступность и глубину геймплея. Элементы исследования обеспечивают высокий уровень вовлеченности.
\end{itemize}

\subsection{Основные особенности игры}
Особенности игры...

\subsection{Описание игры}
Описание игры...

\subsection{Предпосылки создания}
 Предпосылки создания здесь...
%\newpage
\subsection{Платформа}

\begin{table}
    \begin{tabularx}{\textwidth}{|X|X|X|} \hline  
         Требования&  Минимальные& Рекомедуемые\\ \hline  
         ОС&  Windows 8& Windows 10\\ \hline  
         Прцессор&  Intel Core 2 Duo E4400& Intel Core i3-1000NG4\\ \hline  
         ОЗУ&  2GB& 4GB\\ \hline  
         Сводобное место на HDD&  3GB& 4GB\\ \hline  
         Видеокарта&   GeForce 210; RADEON X600  HyperMemory&  GeForce GT 230; Radeon HD 6550D\\ \hline  
         VRAM&  256MB& 512MB\\ \hline  
         DirectX&  9.0& 9.0\\ \hline  
         Управление&  Клавиатура, мышь& Клавиатура, мышь\\ \hline 
    \end{tabularx}
    
    
\end{table}


\newpage
\section{Функциональная спецификация}
\subsection{Принципы игры}
\subsubsection{Суть игрового процесса}
% Суть игрового процесса здесь...

\subsubsection{Ход игры и сюжет}
% Ход игры и сюжет здесь...

\subsection{Физическая модель}
% Описание физической модели здесь...

\subsection{Персонаж игрока}
% Описание персонажа игрока здесь...

\subsection{Элементы игры}
% Элементы игры здесь...

\subsection{Искусственный интеллект}
% Описание искусственного интеллекта здесь...

\subsection{Многопользовательский режим}
% Многопользовательский режим здесь...

\subsection{Интерфейс пользователя}
\subsubsection{Блок-схема}
% Блок-схема интерфейса пользователя здесь...

\subsubsection{Функциональное описание и управление}
% Описание управления здесь...

\subsubsection{Объекты интерфейса пользователя}
% Объекты интерфейса пользователя здесь...

\subsection{Графика и видео}
\subsubsection{Общее описание}
\subsubsection{Двумерная графика и анимация}
\subsubsection{Двумерная графика и анимация}
\subsubsection{Трехмерная графика и анимация}
\subsubsection{Анимационные вставки}
\subsection{Звуки и музыка}
\subsubsection{Общее описание}
\subsubsection{Звуки и звуковые эффекты}
\subsubsection{Музыка}
\subsection{Описание уровней}
\subsubsection{Общее описание дизайна уровней}
\subsubsection{Диаграмма взаимного расположения уровней}
\subsubsection{ График введения новых объектов}
\section{Контакты}

% Общее описание графики и видео здесь...

% Добавьте остальные разделы и текст по мере необходимости...

\end{document}
